\documentclass{article}
\usepackage{amsfonts}
\usepackage[a4paper]{geometry}
\usepackage{alltt}
\usepackage{lmodern}
\usepackage{amssymb}
\usepackage{mathtools}
\usepackage{amsmath}
\usepackage{enumerate}
\usepackage{array}
\usepackage{listings}
\usepackage{fullpage}
\usepackage[parfill]{parskip}
\usepackage[utf8]{inputenc}
\usepackage[ngerman]{babel} 

\title{SE3 Logik Zusammenfassung}
\author{Arne Beer, MN 6489196}

\begin{document}
\maketitle
\section*{Rueckblick Funktionen, Relationen}
    \begin{itemize}
        \item[Funktion(en)] 
            \begin{itemize}
                \item Injektiv
                \item Surjektiv
                \item Bijektiv
                \item $R \leq A x B$ 
            \end{itemize}

        \item[Relation(en)]
            \begin{itemize}
                \item Transitiv $r(x,y)$, $r(y, z)$ $\Longrightarrow r(x,z)$
                \item Reflexiv $r(x,x)$
                \item Symmetrisch $r(y,x)$ ; $r(y,x)$
                \item Aequivalenzrelation, Ordnungsrelation
            \end{itemize}
    \end{itemize}
\section*{Eigenschaften von Prolog}
    \begin{itemize}
        \item . Wird benutzt um einen Befehl zu beenden (equivalent zu ;)
        \item = als Vergleichsoperator
        \item a Kleinbuchstaben sind Konstanten
        \item A Grossbuchstaben sind Variablen
        \item Bei Prolog kann man den Scope einer Variable nicht einschraenken
        \item Sobald eine Variable nicht mehr benoetigt wird, wird sie gekillt.\\
              Dementsprechend gibt es kein globalen Variablen, welche ueber die komplette Programmlaufzeit vorhanden seien koennen.
        \item 'A' von einem ' eingeschlosse Begriffe werden als konstanten interpretiert.
        \item - + werden als Strukturen interpretiert. Also wird aus $4/2$ der Ausdruck $(4,2)$, also die Relation auf 4 und 2. \\
                Dementsprechend ist $/(A,2) = /(5,B)$ wahr, falls $A=5$ und $B=2$.
        \item $=:=$ Dieser Operator beruecksichtigt arithmetische Operationen. Hierdurch koennen normale Rechenoperationen durchgefuehrt werden.
        \item , als logischer UND Operator.
        \item < > werden weiterhin als groesser, kleiner Operator interpretiert. Also liefert $2<3$ ein true.
        \item @< oder @> ueberpruefen die Ordnungrelation zwischen Strukturen. Es gilt Variablen < Zahlen < Konstanten.\\
                Dementsprechend gilt A @< a true.
    \end{itemize}

\section*{Syntaktische Eigenschaften von Prolog}
    \begin{itemize}
        \item Klausel = Fakt | Ziel |
        \item Fakt = Struktur
        \item Fakt = Struktur
        \item Struktur = Name [ ’(’ Term {’,’ Term} ’)’ ]
        \item Term = Konstante | Variable | Strukturen | List
        \item Konstante = Zahl | Name | quoted Name
        \item Name = Kleinbuchstabe gefolgt von beliebig vielen alpha-nums
        \item Quoted Name = beliebiger Name in '' einfachem Apostroph eingeschlossen
        \item Variable = Benannte oder unbenannte Variable sein.
        \item unbenannte Variable = \_name ist eine anonyme Variable
        \item benannte Variable = Name 
    \end{itemize}
\end{document}