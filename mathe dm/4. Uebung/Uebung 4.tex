\documentclass{article}
\usepackage{amsfonts, amsmath,amssymb}
\usepackage[ngerman]{babel}
\usepackage[parfill]{parskip}
\usepackage[latin1]{inputenc}

\title{Mathematik Hausaufgaben zum 15./16. November}
\author{Arne Beer, MN 6489196 \\
 Tim Overath, MN 6440863}

\begin{document}
\maketitle

\section*{1.}
\subsection*{a)}
Es existieren $7^5$ Abbildungen $ g: X \rightarrow Y $ \\
Davon sind $\frac{7!}{2!}=2520  $ injektiv \\
Es gibt $ 7 \cdot 7 \cdot 7\cdot 6 \cdot 5 \cdot =10290$ f�r $ g: X \rightarrow Y $ bei denen g(2), g(3) und g(4) drei verschiedene Elemente sind. 
\subsection*{b)}
Es gibt $\left( \begin{array}{c} 49 \\ 6 \end{array} \right) = 13983816 $ Tips bei Lotto "`6 aus 49 " 

\subsection*{c)}
Es werden die Anzahl der M�glichkeiten der Teilmeingen 997-1000 berrechnet, da nach mindestens 997 Teilmengen gefragt wird.

\[\binom {1000}{997}+\binom {1000}{998}+\binom {1000}{999}+\binom {1000}{1000}=166667501\]



\section*{2.}

\subsection*{a)}

\[x^5 y^{11} in (x+y)^{16} \rightarrow \binom{16}{5}  \]

\[x^3y^5z^2 in (x+y+z)^{10} \rightarrow \frac{10!}{3!\cdot 5! \cdot 2!}\]


\subsection*{b)}

CAPPUCINO:
$ \frac{10!}{3!\cdot 2!} = 302400 $ M�glichkeiten

MANGOLASSI:
$ \frac{10!}{2! \cdot 2!} = 907200 $ M�glichkeiten

SELTERWASSER:
$ \frac{12}{3! \cdot 3! \cdot 2!}  = 6652800$ M�glichkeiten


\subsection*{c)}
Es handelt sich hierbei um ein Ziehen mit Zur�cklegen und ohne Reihenfolge
Demnach gilt: 
\[ \binom{6+10-1}{6}=\binom{15}{6} = 720 \]
Es gibt 720 M�glichkeiten eine Kiste mit 6 Flaschen zusammenzustellen.

\section*{3.}
\subsection*{a)}
Induktionsannahme f�r ein $ n\geq $ 3:
\[ \sum_{i=3}^{n} \left( \begin{array}{c} i \\ i-3 \end{array}  \right) = \left( \begin{array}{c} n+1 \\ 4 \end{array}  \right) \] \\
Induktionsanfang mit n=3: 
\[ \sum_{i=3}^{3} \left( \begin{array}{c} i \\ i-3 \end{array}  \right) = \left( \begin{array}{c} 4 \\ 4 \end{array}  \right) \] 
\[ \Rightarrow  \frac{3!}{3!\cdot 0!}  = \frac{4!}{4!}\]
\[ \Rightarrow  1=1 \text{ Wahre Aussage} \]  \\
Induktionsschritt:
\[ \sum_{i=3}^{n+1} \left( \begin{array}{c} i \\ i-3 \end{array}  \right) = \left( \begin{array}{c} n+2 \\ 4 \end{array}  \right) \] \\
\[\Leftrightarrow \left( \begin{array}{c} n+2 \\ 4 \end{array} \right) = \left( \begin{array}{c} n+1 \\ 4 \end{array} \right) + \left( \begin{array}{c} n+1 \\ n+1-3 \end{array} \right) \] \\
\[\Leftrightarrow \left( \begin{array}{c} n+2 \\ 4 \end{array} \right) = \left( \begin{array}{c} n+1 \\ 4 \end{array} \right) + \frac{(n+1)!}{3!\cdot (n-2)!} \] \\
\[\Leftrightarrow  \left( \begin{array}{c} n+2 \\ 4 \end{array} \right) = \left( \begin{array}{c} n+1 \\ 4 \end{array} \right) + \frac{(n+1)!}{3!\cdot (n+1-3)!} \]
\[\Leftrightarrow  \left( \begin{array}{c} n+2 \\ 4 \end{array} \right) = \left( \begin{array}{c} n+1 \\ 4 \end{array} \right) + \left( \begin{array}{c} n+1 \\ 3 \end{array} \right) \]

Die Aussage ist durch die uns bekannte Rekursionsformel bewiesen. Die Behauptung stimmt demnach.

\section*{4.}

 \subsection*{a)}

Anzahl derjenigen $ k \in \mathbb{N} (1\leq k \leq 2000)$, die weder durch 3,5 oder 7 teilbar sind: \\
Es sei $S={k \in \mathbb{N}:1\leq k\leq 1000} \text{ und } N=|S|=2000$

$A_1=\{k\in S : 3 \mid k\}  \Rightarrow |A_1|=\lfloor \frac{2000}{3}  \rfloor = 666$

$A_2=\{k\in S : 5 \mid k\}  \Rightarrow |A_2|=\lfloor \frac{2000}{5}  \rfloor = 400$

$A_3=\{k\in S : 7 \mid k\}  \Rightarrow |A_3|=\lfloor \frac{2000}{7}  \rfloor = 285$

$A_1\cup A_2=\{k\in S : 15 \mid k\}  \Rightarrow |A_1\cup A_2|=\lfloor \frac{2000}{15}  \rfloor = 133$

$A_2\cup A_3=\{k\in S : 35 \mid k\}  \Rightarrow |A_2\cup A_3|=\lfloor \frac{2000}{35}  \rfloor = 57$

$A_3\cup A_1=\{k\in S : 21 \mid k\}  \Rightarrow |A_3\cup A_1|=\lfloor \frac{2000}{21}  \rfloor = 95$

$A_1\cup A_2 \cup A_3=\{k \in S : 105 \mid k\}  \Rightarrow |A_1|=\lfloor \frac{2000}{105}  \rfloor = 19$

Insgesamt erh�lt man:\\
$ |S\ (A_1\cap A_2 \cap A_3)| = 2000 -(666+400+285)+133+95+57-19=915 $

\subsection*{b)}

Anzahl derjenigen $ k \in \mathbb{N} (1\leq k \leq 1000)$, die weder durch 3, 5, 7 oder 11 teilbar sind: \\
Es sei $S={k \in \mathbb{N}:1\leq k\leq 1000} \text{ und } N=|S|=1000$

$A_1=\{k\in S : 3 \mid k\}  \Rightarrow |A_1|=\lfloor \frac{1000}{3}  \rfloor = 333$

$A_2=\{k\in S : 5 \mid k\}  \Rightarrow |A_2|=\lfloor \frac{1000}{5}  \rfloor = 200$

$A_3=\{k\in S : 7 \mid k\}  \Rightarrow |A_3|=\lfloor \frac{1000}{7}  \rfloor = 142 $

$A_4=\{k\in S : 11 \mid k\}  \Rightarrow |A_3|=\lfloor \frac{1000}{11}  \rfloor = 90 $

$A_1\cup A_2=\{k\in S : 15 \mid k\}  \Rightarrow |A_1\cup A_2|=\lfloor \frac{1000}{15}  \rfloor = 66 $

$A_1\cup A_3=\{k\in S : 21 \mid k\}  \Rightarrow |A_1\cup A_2|=\lfloor \frac{1000}{21}  \rfloor = 47$

$A_1\cup A_4=\{k\in S : 33 \mid k\}  \Rightarrow |A_1\cup A_2|=\lfloor \frac{1000}{33}  \rfloor = 30$

$A_2\cup A_3=\{k\in S : 35 \mid k\}  \Rightarrow |A_1\cup A_2|=\lfloor \frac{1000}{35}  \rfloor = 28$

$A_2\cup A_4=\{k\in S : 55 \mid k\}  \Rightarrow |A_1\cup A_2|=\lfloor \frac{1000}{55}  \rfloor = 18$

$A_3\cup A_4=\{k\in S : 77 \mid k\}  \Rightarrow |A_1\cup A_2|=\lfloor \frac{1000}{77}  \rfloor = 12$


$A_1\cup A_2 \cup A_3=\{k\in S : 105 \mid k\}  \Rightarrow |A_1|=\lfloor \frac{1000}{105}  \rfloor = 9$

$A_1\cup A_2 \cup A_4=\{k\in S : 165 \mid k\}  \Rightarrow |A_1|=\lfloor \frac{1000}{165}  \rfloor = 6$

$A_1\cup A_3 \cup A_4=\{k\in S : 231 \mid k\}  \Rightarrow |A_1|=\lfloor \frac{1000}{231}  \rfloor = 4$

$A_2\cup A_3 \cup A_4=\{k\in S : 385 \mid k\}  \Rightarrow |A_1|=\lfloor \frac{1000}{385}  \rfloor = 2$

$A_1\cup A_2 \cup A_3 \cup A_3=\{k\in S : 1155 \mid k\}  \Rightarrow |A_1|=\lfloor \frac{1000}{1155}  \rfloor = 0$

Insgesamt erh�lt man:\\
$ |S\ (A_1\cap A_2 \cap A_3 \cap A_4)| = 1000 -(333+200+142+90)+66+47+30+28+18+12-(9+6+4+2)+0=415
 $




\end{document}