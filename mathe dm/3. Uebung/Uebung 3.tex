\documentclass{article}
\usepackage{amsfonts, amsmath,amssymb}
\usepackage[ngerman]{babel}
\usepackage[parfill]{parskip}
\usepackage[latin1]{inputenc}

\title{Mathematik Hausaufgaben zum 8./9. November}
\author{Arne Beer, MN 6489196 \\
 Time Overath, MN 6440863}

\begin{document}
\maketitle

\section{Rechenbeispiele}

\subsection{Kongruenzaufgaben}
		1)\[ 177 \not\equiv 18 \ \text{(mod 5), da} \ 177-18=159 \ \text{und} \ 5 \nmid 159  \] 
		2)\[ 177 \equiv -18 \ \text{(mod 5), da} \ 177+18=195 \ \text{und} \ 5 \mid 195 \] 
		3)\[ -89 \not\equiv -12 \ \text{(mod 6), da} \ -89+12=77 \ \text{und} \ 6 \nmid 77 \] 
		4)\[ -123 \not\equiv 33 \ \text{(mod 13), da} \ -123-23=146 \ \text{und} \ 13 \nmid 146 \] 
		5)\[ 39 \equiv -1 \ \text{(mod 40), da} \ 39+1=40 \ \text{und} \ 40 \mid 40 \] 
		6)\[ 77 \equiv 0 \ \text{(mod 11), da} \ 77-0=77 \ \text{und} \ 11 \mid 77 \]
		7)\[ 2^{51} \not\equiv 51 \ \text{(mod 2), da} \ 2^{51}-51=2 \cdot2^{50}-2 \cdot25-1 = 2 		\cdot(2^{50}-25)-1 \] 
		    \[ \Rightarrow \ 2 \nmid 2\cdot(2^{50}-25)-1 \ \text{, weil} \ 2 \mid 2\cdot n \ \text{und} \ 2 \nmid 2\cdot n-1 				  \ \text{, wobei} \  n\in \mathbb{N}  \]  \\
		   	\[ \text{oder} \ 2^{51}-51=2251799813685197 \ \text{und} \ 2\nmid 2251799813685197  \]
		   	
		   			   	
\subsection{Euklidischer Algorithmus}
\[ \text{Bestimmung des ggT(7293,378)}  \]
 		\[ 7293=19\cdot378+111  \]
 		\[ 378=3\cdot111+45 \]
 		\[ 111=2\cdot45+21 \]
 		\[ 45=2\cdot21+3 \]
 		\[ 21=7\cdot3+0 \]
 		\[ \Rightarrow \text{ggT}(7293,216)=3 \]
 		

\subsection{Berechnung von Werten}

		\[ \lceil \sqrt{7} \rceil=3,  \lfloor \sqrt{7} \rfloor=2,  \lceil 7,1 \rceil=8 \lfloor 7,1 \rfloor=7 \]
		\[ \lceil -7,1 \rceil=-7, \lfloor -7,1 \rfloor=-8 \lceil -7 \rceil=-7, \lfloor -7 \rfloor=-7 \]

\section{Beweise}

\subsection{Regel (2)}

Beweise:				\[ \text{Aus} \ b_1\mid a_1 \ \text{und} \ b_2 \mid a_2 \ \text{folgt} \ b_1\cdot b_2 \mid a_1\cdot a_2   \]
								\[b_1\mid a_1 \ \text{und} \ b_2\mid a_2 \ \text{ist equivalent zu} \ a_1=b_1\cdot n_1 \ \text{und} \ a_2=b_2\cdot n_2 \]
								\[\Rightarrow a_1\cdot a_2=b_1\cdot n_1 \cdot b_2\cdot n_2 \]
							\[ \text{Eingesetzt in die Behauptung}\ b_1\cdot b_2 \mid a_1\cdot a_2 \ \text{ ergibt sich:} \ b_1\cdot b_2 \mid b_1\cdot b_2 \cdot n_1 \cdot n_2 \]
							\[ \text{Somit ist die Aussage bewiesen}  \]
								
								


\subsection{Regel (3)}
Beweise: 				\[ \text{gilt} \ c\cdot b \mid c\cdot a \ \text{, dann gilt auch} \ b\mid a \ \text{f�r} \ c\neq 0 \]
Allgemeine Definition von Teilern: \[ a=n\cdot b \ \text{wobei} \ a,n,b\in\mathbb{Z} \]
								\[\Rightarrow \text{f�r} \ c\cdot b \mid c\cdot a \ \text{gilt} \ c\cdot a = n\cdot c \cdot b \]
								\[\Rightarrow a=n\cdot b \ \text{was equivalent zur Aussage} \ b\mid a \ \text{ist}  \]
								
								
								
\subsection{Regel (4)}
Zu beweisen: 		\[ \text{Aus} \ b\mid a_1 \ \text{und} \ b\mid a_2 \ \text{folgt} \ b\mid c_1\cdot a_1 + c_2\cdot a_2 \ \text{f�r beliebige} \ c_1, c_2 \in\mathbb{Z}  \]
Beweis:					\[ \text{Es gilt:} \ b\mid a_1 \Leftrightarrow a_1=n_1\cdot b\ \text{und} \ b\mid a_2 \Leftrightarrow a_2=n_2\cdot b \]
 								\[ \text{Eingesetzt in die Behauptung}\ b\mid a_1\cdot c_1 + a_2 \cdot c_2 \ \text{ergibt sich} \ 								  b\mid b\cdot (c_1\cdot n_1+c_2\cdot n_2) \]
								\[ \text{Somit ist die Aussage bewiesen}  \]
								


\section{Aufgabe 3}

\subsection{Beweis durch Vollst�ndige Induktion}

Induktionsannahme: \[ 3\mid (n^3+2\cdot n) \]
Induktionsanfang: \[ 3\mid (1^3+2\cdot1) \ \text{wahre Aussage}  \]
Induktionsschritt:\[ 3\mid (n+1)^3+2\cdot(n+1) = \]
									\[ 3\mid n^3+3n^2+3n+1+2n+2 \] \\
									\\ \
Durch Anwendung der Induktionsannahme folgt: \[ 3\mid n^3+2\cdot n \  \] 
									\[ \text{und} \ 3\mid 3\cdot(n^2+n+1)  \]
									\[ \Rightarrow \ 3\mid n^3+2n+3n^2+3n+3 \]

\subsection{b)}

F�r jedes $n \in N \ 2^n \cdot 2^n $ - Schrachbrett muss es einen Teiler 3 geben bei dem der Rest 1 bleibt.
D.h.  \ \[ 2^1 \cdot 2^1 = 4 \] \ und 3 teilt 4 mit dem Rest 1 \\
\\
Induktionsannahme: \[3 \mid  2^n \cdot 2^n  \text{ mit Rest 1 } \]
Induktionsanfang: \[3 \mid  2^1 \cdot 2^1 \text{ mit Rest 1 }\]
Induktionsschritt:
\[3 \mid 2^{n+1} \cdot 2^{n+1}  \text{ mit Rest 1 }\]
\[3 \mid 2 \cdot 2^n \cdot 2^n \cdot 2 \text{ mit Rest 1 }\]
\[3 \mid 2^n \cdot 2^n \cdot 4 \text{ mit Rest 1 }\]
Anwendung der Annahme:
\[3 \mid 2^n \cdot 2^n  \text{ mit Rest }1\]
\[3 \mid 4 \text{ mit Rest 1}\]
Dadurch ist die Aussage bewiesen.

\section{Funktionen}

\subsection{\[ \mathbb{Q}\times\mathbb{Q} = \mathbb{Q}\times\mathbb{Q}\times\mathbb{Q}  \]}
\[g(x,y)=(xy^2,xy^2-3x,(x^2-2)y) \]
F�r Injektivit�t muss gelten: \[ f(x,y)\neq f(a,b) \ \text{wobei} \ x,y \neq a,b \ \ \text{und} \ x,y \in\mathbb{Q} \]
Beweis:					
					\begin{equation} xy^2=ab^2 \end{equation}
					\begin{equation} xy^2-3x=ab^2-3a  \end{equation}
					\begin{equation} x^2y-2y=a^2b-2b  \end{equation}
(2) in (3)\begin{equation}  \ xy^2+x^2y-3x-2y=ab^2+a^2b-3a-2b \ \end{equation}
(1) in (2) in (3) \[ 2\cdot xy^2+x^2y-3x-2y=2\cdot ab^2+a^2b-3a-2b\]				
					\begin{equation} xy^2+x^2y-3x-2y-ab^2+3a+2b=a^2b \end{equation}
(5) in (4)\[ xy^2+x^2y-3x-2y=ab^2+xy^2+x^2y-3x-2y-ab^2-3a-2b+3a+2b \]				
					\[\Rightarrow xy^2=ab^2 \ \text{steht im Wiederspruch mit} \ x,y\neq a,b  \]
					\[ \Rightarrow \text{Die Funktion ist Injektiv} \]
								
								


\subsection{\[ \mathbb{Z} = \mathbb{Z}\times\mathbb{Z}  \]}
\[ h(z)=(z+2,z-1)  \]
F�r Surjektivit�t muss gelten:\[ \text{zu jedem Tupel (b,c), wobei} \ (b,c)\in\mathbb{Z} \ \text{, ist mindestens ein a, wobei} \ a\in\mathbb{Z} \ \text{, definiert mit} \ f(a)=(b,c)  \]
Beweis: \\
Darstellung des Tupels (2,2)
\[ 2=z+2 \text{ und } 2=z-1 \]
\[ \Rightarrow z+2=z-1 \]
\[\Rightarrow 2=-1  \]
\[ \text{Die Aussage ist falsch und somit ist Funktion nicht Surjektiv, da das Tupel nicht abgebildet werden kann.}  \]


\end{document}