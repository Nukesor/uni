\documentclass{article}
\usepackage{amsfonts}
\usepackage[ngerman]{babel}
\usepackage[a4paper]{geometry}
\usepackage{amsfonts}
\usepackage{alltt}
\usepackage{amsmath,amssymb}
\usepackage[parfill]{parskip}
\usepackage[latin1]{inputenc}

\title{Mathematik Hausaufgaben zum 30. November}
\author{Arne Beer, MN 6489196 \\
 Tim Overath, MN 6440863}

\begin{document}
\maketitle

\section*{6.1}

\subsection*{a)}
\( A = \begin{pmatrix} 2&0&1 \\ 1&0&-1 \\7&6&3 \\ -1&2&4 \end{pmatrix}
    ,B=\begin{pmatrix} 3&2&-1 \\ 1&0&2 \\1&1&0 \end{pmatrix} 
    ,C= \begin{pmatrix} 1 &2&-2 \end{pmatrix}
    ,D= \begin{pmatrix} 2\\3\\-2 \end{pmatrix} \)
    
Folgende der aufgelisteten Produkte sin definierbar. AB, AD, BB, CD, DC

\[AB= \begin{pmatrix} 7&5&-2 \\ 2& 1&-1 \\30&17&5 \\ 3&2&5& \end{pmatrix}, AD= \begin{pmatrix} 
 2\\4\\26\\-4 \end{pmatrix}, BB= \begin{pmatrix} 10&5&1\\5&4&-1\\4&2&1 \end{pmatrix}, CD = \begin{pmatrix} 12 \end{pmatrix}, DC= \begin{pmatrix} 3&4&-4 \\3&6&-6 \\-2&-4&4 \end{pmatrix}\]


\section*{b)}

$ AB_{3;2}=15 $ \\
Die vierte Spalte von AB lautet: $ \begin{pmatrix} 13\\8\\3\\23 \end{pmatrix} $

\section*{6.2}

Damit die G�ltigkeit des Distributivgesetztes mit den vorliegenden Matrizen bewiesen wird muss gelten:
$A(B_1+B_2)=AB_1+AB_2$ \\

\[A\cdot (B_1+B_2)=A\cdot \begin{pmatrix} 2&0 \\ 6&8 \end{pmatrix} = \begin{pmatrix} 52&56\\12&-8\\28&16 \end{pmatrix} \]

\[AB_1\cdot AB_2 = \begin{pmatrix} 26&52\\6&12\\14&28 \end{pmatrix}+\begin{pmatrix} 16&4\\6&-20\\14&-12 \end{pmatrix}=\begin{pmatrix} 52&56\\12&-8\\28&16 \end{pmatrix}\]
Die G�ltigkeit der urspr�nglichen Aussage ist somit auf diese Matrizen bezogen bewiesen. 


\subsection*{b)}

Best�tigen der Gleichung $(AB)^T=B^TA^T$ anhand der vorliegenden Matrizen

\[A^T =\begin{pmatrix} 1&2\\3&6 \end{pmatrix} B^T= \begin{pmatrix} 2&3\\-1&2\\5&4 \end{pmatrix}\]

\[B^T\cdot A^T = \begin{pmatrix} 11&22\\5&10\\17&34 \end{pmatrix}\]

\[(AB)^T =  \begin{pmatrix} 11&5&17\\22&10&34 \end{pmatrix}^T = \begin{pmatrix} 11&22\\5&10\\17&34 \end{pmatrix}\]

\subsection*{c)}

Wie man anhand der Umgeformten Matrizen $ A^T \text{ und } B^T $ erkennen kann, ist eine Multiplikation hier nicht mehr m�glich. Aus dem urspr�nglichen Aufbau $ A =m \times n \text{ und } B=n\times p $ wird $ A^T =n \times m \text{ und } B^T = p\times n $, wodurch die Multiplikation nicht mehr m�glich ist. Ein Ausnahmefall ist hier, wenn 2 quadratische Matrizen miteinander multipliziert werden, da sie ihre urspr�ngliche Form beibehalten. 

\section*{6.3}
A sei eine $m \times n$ - Matrix, $B_2$ und $B_2$ seien $n \times p$ - Matrizen. Beweisen Sie die G�ltigkeit des
Distributivgesetzes $A(B_1+ B_2) = AB_1+ AB_2$

\[ A=(a_{ij})_{\substack {i=1,\cdots,m \\ j=1,\cdots,n}},
	B_1=(b_{jk})_{\substack{j=1,\cdots,n \\ k=1,\cdots,p}},
	B_2=(b'_{jk})_{\substack{j=1,\cdots,n \\ k=1,\cdots,p}} \]
	
\[ B_1+B_2= (b_{jk}+b'_{jk})_{\substack{i=1,\cdots,m \\ j=1,\cdots,n}} \]

\[A(B_1+B_2)=\left(\sum_{j=1}^{n}
		a_{ij}\cdot (b_{jk}+b'_{jk})\right)_
				{\substack{i=1,\cdots,m \\ k=1,\cdots,p}}\]

\[A(B_1+B_2)=\left(\sum_{j=1}^{n}
		(a_{ij}\cdot b_{jk}+a_{ij} \cdot b'_{jk})\right)_
				{\substack{i=1,\cdots,m \\ k=1,\cdots,p}}\]

\[A(B_1+B_2)=\left(\sum_{j=1}^{n}
		a_{ij}\cdot b_{jk} + \sum_{j=1}^{n}
		a_{ij}\cdot b'_{jk} \right)_
				{\substack{i=1,\cdots,m \\ k=1,\cdots,p}}\]
				
\[A(B_1+B_2) = \left(\sum_{j=1}^{n}
		a_{ij}\cdot b_{jk}\right)_
				{\substack{i=1,\cdots,m \\ k=1,\cdots,p}}
+ \left(\sum_{j=1}^{n}
		a_{ij}\cdot b'_{jk}\right)_
				{\substack{i=1,\cdots,m \\ k=1,\cdots,p}} \]
				
\[A(B_1+B_2) = AB_1+AB_2\]
				
				

\section*{6.4}

\subsection*{a)}
Zu beweisende Aussage: F�r jede Abbildung \\
$ f: A \rightarrow B $ und jedes $ B \subseteq B' $ gilt $ f(f^{-1}(B')) \subseteq B' $

Es ist zu zeigen das f�r ein beliebiges $x \in f(f^-1(B')) $ gilt $x \in B'$ .  \\
Es gilt $x \in B'$, $x' \in f^-1(B')$. Es gibt ein y f�r das gilt $f(y)\subseteq B'$



\end{document}	