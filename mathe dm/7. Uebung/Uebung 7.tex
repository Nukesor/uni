\documentclass{article}
\usepackage{amsfonts}
\usepackage[ngerman]{babel}
\usepackage[a4paper]{geometry}
\usepackage{amsfonts}
\usepackage{alltt}
\usepackage{amsmath,amssymb}
\usepackage[parfill]{parskip}
\usepackage[latin1]{inputenc}

\title{Mathematik Hausaufgaben zum 7. Dezember}
\author{Arne Beer, MN 6489196 \\
 Tim Overath, MN 6440863}

\begin{document}
\maketitle

\section*{7.1}
\subsection*{a)}
Zuerst wird �berpr�ft ob 473 und 2413 teilerfremd sind. Hierf�r muss gelten \\ $ggT(2413,473)=1$
\[

2413=473\cdot5+48 \\
473=48\cdot9+41 \\
48=41+7 \\
41=7\cdot5+6 \\
7=6\cdot1+1\]

Hiermit ist bewiesen dass 2413 und 473 Teilerfremd sind. 

Das Inverse wird durch r�ckw�rtiges Einsetzen in den Euklidischen Algorithmus errechnen.
\[

1=7-6\cdot1\\
1=6\cdot7-41\\
1=7 - 6 \cdot 1
1=7 - (41 - 7 \cdot 5)\\
1=6 \cdot 7 - 41\\
1=6 \cdot (48 - 41 \cdot 1) - 41\\
1=6 \cdot 48 - 7 \cdot 41\\
1=6 \cdot 48 - 7 \cdot (473 - 48 \cdot 9)
1=69 \cdot 48 - 7 \cdot 473\\
1=69 \cdot (2413 - 473 \cdot 5) - 7 \cdot 473\\
1=-352 \cdot 473+69 \cdot 2413\\
\]

\subsection*{b)}

Zuerst wird �berpr�ft ob 473 und 2413 teilerfremd sind. Hierf�r muss gelten \\
$ggT(2413,1672)=1$
\[

2413=1672\cdot1+741\\
1672=741\cdot2+190\\
741=190\cdot3+171\\
190=171\cdot1+19\\
171=19\cdot9+0\\
\]
Es gilt ggT(2413,1672) = 19, woraus folgt, dass 1672 nicht invertierbar ist.

\section*{7.2}

Der Satz von Fermat besagt dass f�r eine Primzahl p und eine nat�rliche Zahl n, bei der  $p\nmid n$, die Aussage $n^{p-1}\equiv 1$ gilt. 
Man kann also annehmen dass in $\mathbb{Z}_{19}$ 
$3^{18}=1$ gilt.
Nun kann man $3^{1000}$ so umformen, dass ein m�glichst gro�er Teil durch Umformung in 1 wegf�llt. 

$3^{1000}=(3^{18})^{55}\cdot 3^{10}=3^{10}$

$\Leftrightarrow 3\cdot (3^3)^3=3\cdot(8)^3 $

$\Leftrightarrow3\cdot(8)^3=16$

\section*{7.3}

\begin{itemize}

\item a) $\pi = (1,7,6)(2,10,8,5,11,13)(3,4)(9,12)$

\item b) $\pi = (1,6) \circ (1,7) \circ (2,12) \circ (2,11) \circ (2,5) \circ (2,8) \circ (2,10) \circ (3,4) \circ (9,12)$

\item c) sign $\pi = -1$ (ungerade)


\end{itemize}

\section*{7.4}

\subsection*{a)}
Da die Berrechnung der Elemente der Berrechnung aller geordneter M�glichkeiten gleicht, kann man die Anzahl der Elemente einfach �ber die Multiplikation der einzelnen Mengengr��en errechnen. 

$3\cdot 5 \cdot 2= 30$

\subsection*{b)}

Die Anzahl der m�glichen tern�ren Relationen errechnet man, indem man die Anordnungsm�glichkeiten der Kreuzprodukte untereinander berrechnet. Bei 3 Mengen w�ren es folglich 3!=6 m�gliche Anordnungsm�glichkeiten, solange gilt, dass keine Menge doppelt vorkommt, also \\
$A\neq B\neq C$. Wenn jedoch jede Menge beliebig oft vorkommen darf, lassen sich $3^3$ m�gliche tern�re Relationen bilden. 


\end{document}