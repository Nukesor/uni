\documentclass{article}
\usepackage{amsfonts}
\usepackage[a4paper]{geometry}
\usepackage{alltt}
\usepackage{lmodern}
\usepackage{amssymb}
\usepackage{mathtools}
\usepackage{amsmath}
\usepackage{enumerate}
\usepackage{array}
\usepackage{listings}
\usepackage{fullpage}
\usepackage{color}
\usepackage[parfill]{parskip}
\usepackage[utf8]{inputenc}
\usepackage[ngerman]{babel} 
\usepackage[T1]{fontenc}

\usepackage{tikz}

\title{AD-Übung zum 19. November}
\author{Arne Beer, MN 6489196\\
Merve Yilmaz, MN 6414978\\
Sascha Schulz, MN 6434677}

\begin{document}
\maketitle

\begin{enumerate}[\bfseries1.]
    \item
        \begin{enumerate}
            \item
                $h(k) = k$ mod 11 $\Rightarrow$  11$\mathbb{N}$ + 10 auf
                Index10, 
                \\da $11 \cdot n \text{ mod } 11 = 0$, $n \in \mathbb{N}$ und nach
                einer Offset verschiebung durch die Addition der 10. Index erreicht ist.
            \item
                $h(k) = 2k$ mod 11  $\Rightarrow$ 11$\mathbb{N}$ + 5 auf
                Index10, 
                \\da $2 \dot (an+b) \text{ mod } 11 = 10$ mit $n \in
                \mathbb{N}$ 
                \\gefordert ist. Nach anwenden des
                Distributivgesetztes: 
                \\$(2an + 2b) \text{ mod } 11 = 10$. 
                \\Da 11
                eine Primzahl ist, lässt sich diese nich durch die
                Multiplikation von natürlichen Zahlen erzeugen, folglich sollte
                \\$2an \text{ mod } 11 = 0$ 
                \\sein, sodass die Offset-Verschiebung
                wie in Aufgabe 1 durch die Addition vorgenommen wird. 
            \item
                $h(k) = k^2 + 10$ mod 11 $\Rightarrow$ 11$\mathbb{N}+0$ auf
                Index10,
                \\da für alle k als Vielfache von 11 gilt: 
                \\$k^2 \text{ mod }11 = 0$.
                \\ Wobei durch die anschließende Addition von 10 jeder Vielfache
                von 11 auf Index10 abgebildet wird. $b=0$ ist erforderlich, da
                auch der Index 0 auf Index10 abgebildet wird.
            \item
                $h(k) = 3^k -1$ mod 11 $\Rightarrow \emptyset$ auf Index10,
                \\ Damit $h(k)$ auf Index10 abbildet, ist
                erforderlich, dass $3^k$ einen Vielfachen von 11 bilden.
                Dies würde bedeuten, dass eine Zahl, welche in der
                Primzahlzerlegung aus $k$-Fach 3en besteht zudem durch die
                Primzal 11 teilbar ist. Da 11 und 3 beides Primzahlen sind ist
                dies schlicht nicht möglich.
        \end{enumerate}

    \item
         Zu Beginn wird $n!$ mit $n^{n}$ verglichen.
        \[
                \frac{n \cdot n \cdot n \cdot \text{...} \cdot n \cdot n}{n \cdot (n-1) \cdot (n-2) \cdot \text{...} \cdot 2 \cdot 1}
        \]
        Es wird deutlich, dass $n!$ asymptotisch langsamer 
        wächst als $n^{n}$. Anschließend vergleichen wir $n!$
        mit $\left(\frac{n}{2}\right)^{\frac{n}{2}}$.
        \[
                \frac{n \cdot (n-1) \cdot (n-2) \cdot ... \cdot (n- \frac{n}{2}) \cdot (n - \frac{n}{2} - 1) \cdot ... \cdot 2 \cdot 1}{\frac{n}{2} \cdot \frac{n}{2} \cdot \frac{n}{2} \cdot ... \frac{n}{2} \cdot 1 \cdot ... \cdot 1 \cdot 1}
        \]
        Es wird deutlich, dass $n!$ asymptotisch schneller 
        wächst als $\left(\frac{n}{2}\right)^{\frac{n}{2}}$.
        
        Aufgrund dieser Feststellungen wird nun der Logarithmus
        von $\left(\frac{n}{2}\right)^{\frac{n}{2}}$ und $n^{n}$ 
        gebildet und mit dem von $n!$ verglichen.

        \begin{alignat*}{2}
                \log\left(\left(\frac{n}{2}\right)^{\frac{n}{2}}\right) &=& \frac{n}{2} \log\left(\frac{n}{2}\right) \\
                &=& \frac{1}{2}n \log\left(\frac{1}{2}n\right) \\
                \log(n^{n}) &=& n \log n
        \end{alignat*}
        Damit ist klar, dass die Logarithmen von 
        $\left(\frac{n}{2}\right)^{\frac{n}{2}}$ und $n^{n}$ beide
        in $\theta(n \log n)$ sind. Aus unserem obigen Vergleich wissen 
        wir, dass $n!$ schneller als $\left(\frac{n}{2}\right)^{\frac{n}{2}}$ 
        und langsamer als $n^{n}$ wächst. Daraus ergibt sich:
        \[
                \frac{1}{2}n\log(\frac{1}{2}n) \in \theta(n \log n) \leq \log(n!) \leq n \log n \in \theta(n \log n)
        \]
        Da $\log(n!)$ asymptotisch sowohl schneller als auch langsamer 
        als $n \log n$ wachsen muss, liegt $\log(n!)$ damit  folgerichtig in $\theta(n \log n)$.
    \item
        \begin{enumerate}
             \item $T(n)=2\cdot T(\frac{n}{2})+a\cdot n$ \\
                      $= \Theta (n\cdot \log{n}$) \\
                      Die Bestimmung des Medians betraegt $O(n)$. Die zu sortierenden Elemente werden in zwei Teile aufgeteilt unter Verwendung des Medians.
             \item Diese Variante wird in der Praxis nicht benutzt, da die
            Konstante a zum Auffinden des Medians sehr groß ist.
                
            \item \textcolor{red}{TODO: Kein Loesungsansatz gefunden.}
        \end{enumerate}
    
    \item
        \begin{enumerate}
            \item
                Wir stellen uns einen binären Entscheidungsbaum vor. Ist das
                Ergebnis des Münzwurfs eine 0, wählen wir den linken Ast,
                alternativ bei 1 den rechten Ast. Dies führen wir fort, bis zur 
                Ebene $k$, in dieser Befinden sich $2^k$ Blätter, wobei jedes
                Blatt einem Element unseres Arrays entspricht.
                
                Der Baum hat somit eine Tiefe von $k = log_2(n)$, mit $n=2^k$.
                Die Anzahl der nötigen Münzwürfe liegt folglich bei $O(log(n))$,
                da mit jedem Wurf eine Ebene tiefer erreicht wird. Da die Anzahl
                der Ebenen endlich ist ($k$ Stück), terminiert dies stets. Da
                genau ein Pfad im Baum zu genau einem Element im Array führt,
                besitzt jedes Element die exakt gleiche Wahrscheinlichkeit
                gezogen zu werden.
            
            \item
                Wir wandeln den zuvor vollen binären Entscheidungsbaum in einen
                vollständigen Baum um, sodass weiterhin jedes Blatt des Baumes
                ein Element des Array repräsentiert.
                
                Dies hat zur Folge, das es Pfade geben kann, für die genau ein
                Münzwurf weniger für die Entscheidung benötigt wird, da sie sich
                nicht auf der untersten (unvollständigen) Ebene befinden,
                sondern auf der vorherigen.
                
                Die maximale Tiefe des Baumes und somit die Anzahl der
                benötigten Münzwürfe bleibt unverändert $O(log(n))$, da maximal 
                $\lceil log_2(n)$ Würfe benötigt werden und das Ceiling in der
                Landau-Notation nicht berücksichtigt werden muss, da dies die
                Addition von $x \in [0,1]$ bedeutet, wobei das Ceiling erst für
                3+ Elemente notwendig ist (da es sich zuvor eh um einen vollen
                Baum handelt). In diesen Fällen ist bereits $log_2(n) > 1$.

            \item  
                Um zu gewährleisten, dass für jeden Eintrag exakt die selbe
                Wahrscheinlichkeit ($\frac{1}{n}$) gegen ist, müssen zur Wahr
                eines Elementes exakt gleich viele Münzwürfe benötigt werden
                (Bei der asymptotischen Betrachtung ist es jedoch fraglich, ob
                nicht bereits die zuvor skizzierte Lösung 'exakte'
                Wahrscheinlichkeiten liefert im Sinne von +/- 1 Münzwurf auf
                $O(log(n))$ Würfe).
                
                Um dies zu gewährleisten gehen wir initial von einem
                vollständigen binären Baum aus, den wir anschließend
                modifizieren.
                
                Bei einem vollständigen Baum befinden sich in der untersten
                Ebene $2^{log(n)}$ Blätter. Von links an wird jedem Element ein
                Blatt zugeordnet, bis alle Elemente injektiv abgebilet sind.
                
                Die nun potentiell übrig gebliebenden Blätter werden rekursiv
                entfernt, sofern auf zwei Blätter eines Elternknotens nicht
                abgebildet wurde. Diese `toten Blätter` verweisen auf den
                Wurzelknoten - es muss erneut von vorn mit Münzwürfen begonnen
                werden, um die Gleichverteilung der Wahrscheinlichkeit unter
                keinen Umständen zu gefährden, potentiell auf Kosten der
                Laufzeit - bis kein totes Blatt mehr erreicht wurde.
                
                Die maximale Anzahl dieser `toten Blätter
                beträgt im Worst Case $log_2(n) - 1$ (es wird genau 1 Blatt
                des Teilbaumes verwendet, welches vom Wurzelelement nach rechts
                geht, ab da befindet sich auf jeder Ebene 1 totes Blatt), im
                Best Case 0 (vollständiger Baum).
                
                Ich erhoffe mir, dass im Erwartungswert auf Grund von
                Wahrscheinlichkeitsrechnung und Kombinatorik -  welcher ich in
                diesem Augenblick nicht fähig bin - $O(log(n))$ Münzwüfe
                benötigt werden um ein Element zu bestimmen, jedoch zufällig -
                durch das Erreichen eines toten Blattes, diese Anzahl schwanken
                kann.
                
                Zugegeben, eine kreative Lösung. Finde ich aber interessanter
                als in der Literatur nach fremden Ideen zu suchen um schlicht zu
                beweisen, dass ich motiviert bin, Zeit aufzuwenden um dann eine
                Fremdlösung zu präsentieren.

        \end{enumerate}
    
    \item
        \begin{enumerate}
            \item BerechnungsbaUm der durchgeführten Vergleichsabläufe für
                 Eingabe [a,b,c]
                 
             \begin{tikzpicture}[%
                    level 1/.style={sibling distance=40mm},
                    level 2/.style={sibling distance=15mm},
                    level 3/.style={sibling distance=15mm}
                    ]
    \node {b<=c}
    child {
        node{a<=b}
        child{
            node{abc}
            edge from parent
                node[left] {true}
        }
        child{
            node{a<=c}
            child{
                node{bac}
                edge from parent
                    node[left] {true}
            }
            child{
                node{bca}
                edge from parent
                    node[right] {false}
            }
            edge from parent
                node[right] {false}
        }
        edge from parent
            node[left] {true}
    }
    child{
        node{a<=c}
        child{
            node{acb}
            edge from parent
                node[left] {true}
        }
        child{
            node{a<=b}
            child{
                node{cab}
                edge from parent
                    node[left] {true}
            }
            child{
                node{cba}
                edge from parent
                    node[right] {false}
            }
            edge from parent
                node[right] {false}
        }
        edge from parent
            node[right] {false}
    };
\end{tikzpicture}
            
            \item Anzahl der Blätter bei Eingabe von [a, b, c, d]:    
                    $\quad 4!$
            
                    Anzahl der Blätter bei Eingabe von [a, b, c, d, \ldots, w,
                    x, y, z]: 
                    $\quad 24! $
        \end{enumerate}
    \item
    Zunächst wird mit Hilfe von ,,Median of Medians'' \ das k't-grösste Element
    gefunden und alle Elemente, die größer sind als $k$, aus der zu sortierenden
    Liste entfernt. Wir setzen vorraus, dass dies eine Laufzeit von
    $\mathcal{O}(n)$ hat.

    Danach wird mit einem deterministischen Quicksort-Verfahren sortiert, bspw.
    aus a) in $\mathcal{O}(n)$. Zusammen also läge man damit in $\mathcal{O}(n+k\log k)$.
    Hierfür gilt:

    \[ \frac{n+k \log k}{n \log k} = \frac{1}{\log k} + \frac{k}{n} \]

    Aus $k \ll n$ folgt, dass der letzte Bruch annähernd $0$ ist, und $\frac{1}{\log k}$
    geht für $k \rightarrow \infty$ gegen $0$. Somit ist gezeigt, dass
    $(n+k \log k)$ für $k \ll n$ in $\mathcal{O}(n \log k)$ liegt.


    \end{enumerate}
\end{document}  