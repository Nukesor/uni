\documentclass{article}
\usepackage[a4paper]{geometry}
\usepackage{amsfonts}
\usepackage[parfill]{parskip}
\usepackage[latin1]{inputenc}
\title{RS-Übung 3}
\author{Arne Beer (MN 6489196), \\
Rafael Epplee (MN 6269560), \\
Julian Polatynski (MN 6424884)}

\begin{document}

\maketitl e

\section{Subtraktion mit Komplementen}
\subsection*{(a)}
Das Zweierkomplement von 532 ist 9468. Demnach wäre die Rechnung
\[ 1385-532 = 1385
+9468
= 10853\]

Da wir nur mit 4 Stellen rechnen, ist das Ergebnis 853.

\subsection*{(b)}
Komplement von 687 ist 9313.
\[372 - 687 = -315\]g
\[\Leftrightarrow 372 + 9313 = 9685\]

Das Ergebnis ist wieder ein Zweierkomplement.

\subsection*{(c)}
Zunächst werden die Zahlen aus dem Dezimalsystem in das Binärsystem umgewandelt:
\[\begin{tabular}{l | r}
$1385_{10}$ & $010101101001_2$ \\
$532_{10}$ & $001000010100_2$ \\
$K_2(532)$ & $110111101100_2$ \\
\end{tabular}\]

Dann ergibt sich folgende Rechnung:

\[\begin{tabular}{r | r}
$010101101001$ & $1385_{10}$\\
$+110111101100$ & $+K_2(532)$\\ \hline
$1001101010101$ & $=853$
\end{tabular}\]

\subsection*{(d)}
\[ 372_{10}=000101110100_2 \]
\[ 687_{10}=001010101110_2 \]
\[ K_2(001010101110_2)=110101010001_2 \]

\[\begin{tabular}{r}
000101110100\\
+110101010001\\
\hline
$111011000101$\\
\end{tabular}\]

Es entsteht kein Übertrag, woraus folgt, dass es sich hierbei um eine negative Zahl handelt. Bildet man nun das 2-Komplement, erhält man das Ergebnis der ursprünglichen Rechnung.

\[ K_2(111011000101)_2=000100111011_2 = 315_{10} \]

Da das Ergebnis ursprünglich eine negative Zahl ist, ist das tatsächliche Ergebnis $-315$.

\section{Gleitkommazahlen normalisieren}
\[(a)\ (6,9242\ |\ 4)_{10}\]
\[(b)\ (-1,100101\ |\ -10)_2\]
\[(c)\ (-2,D4A\ |\ B)_{16}\]

\section{IEEE754}
\[(a)\ 0\ |\ 1000\ 0101\ |\ 000\ 0000\ 0000\ 0000\ 0001\ 1011 \]
\[(b)\ 1\ |\ 1000\ 0110\ |\ 000\ 0000\ 0000\ 0101\ 0100\ 0101 \]

Die \(|\) - Zeichen dienen hier nur der Übersicht.

\section*{3.4 Gleitkomma-Addition}
\subsection*{(a)}
Schritt 1: Anpassen des niedrigeren Exponenten an den höheren:
\[8,626 \cdot 10^5 = 0,08626 \cdot 10^7 \]
Schritt 2: Addition der Mantissen:
\[0,08626 + 9,9442 = 10,03046\]
Schritt 3: Normalisieren des Ergebnisses:
\[10,03046 \cdot 10^7 = 1,003046 \cdot 10^8\]
Schritt 4: Runden des normalisierten Ergebnisses:
\[1,003046 \approx 1,0030 \]

\subsection*{(b)}
\[ 8,626 \cdot 10^5 +9,9442 \cdot 10^7 \]
\[ = 0,08626 \cdot 10^7 +9,9442 \cdot 10^7 \]
\[ \approx \ 0,0862 \cdot 10^7+9,9442 \cdot 10^7 \]
\[ = (0,0862 + 9,9442) \cdot 10^7 = 1,0030 \cdot 10^7 \]


\section{Gleitkomma-Multiplikation}
\[ 5,6538 \cdot 10^7 \cdot 3,1415 \cdot 10^4 \]
\[ = 5,6538 \cdot 3,1516 \cdot 10^{7+4} \]
\[ = 17,7614127 \cdot 10^{11}=1,7761\cdot 10^{12} \]

\end{document}