\documentclass{article}
\usepackage{amsfonts,amsmath}
\usepackage[latin1]{inputenc}
\usepackage[parfill]{parskip}

\title{RS-�bung 2}
\author{Arne Beer (MN 6489196), \\
Rafael Epplee (MN 6269560), \\
Julian Polatynski (MN 6424884)}

\begin{document}
\maketitle

\section{Time Stamp Counter}
\subsection{�berlauf}
Bei einer Gr��e von 64 Bit hat das Register Zahlen bis zu \(2^{64}-1\) aufnehmen.

Um herauszufinden, wie lange es dauern w�rde, bis diese Zahl �berschritten ist, teilen wir zun�chst durch die Anzahl der Takte pro Sekunde, dann durch Sekunden pro Minute, Minuten pro Stunde, Stunden pro Tag, Tage pro Jahr.

\[ \frac{2^{64}-1}{3,2 \cdot 10^9}\cdot s = 5,76461 \cdot 10^9s\]

Es w�rde also erst nach ungef�hr 182,6695 Jahren zum �berlauf des Registers kommen.

\subsection{Eventuelle Nachteile}
Die Taktzahl eines Prozessors bleibt nicht immer gleich. Sie wird beispielsweise heruntergefahren, um Energie zu sparen, oder auch durch den Benutzer �ber den Standardwert hinaus gesteigert. Bei einer �nderung der Taktrate w�re dann auch die gemessene Zeit ung�ltig, da sich das Verh�ltnis der Takte zu den Sekunden ver�ndert h�tte.
Ein weiteres Problem bei dieser Zeitmessung ist, dass die Zeit bei unterschiedlich getakten Prozessoren unterschiedlich schnell vergeht. D.h. Bei einem Prozessor der beispielsweise mit 1,4 GHZ getaktet ist (z.B Galaxy S Plus) vergeht die Zeit etwa doppelt so langsam. 

\section{Umwandlung von Dezimalzahlen}
\subsection{53}
\begin{tabular}{|rl|}
\hline
35 & hexadezimal \\
65 & oktal \\
110101 & bin�r \\
\hline
53 & dezimal \\
\hline
\end{tabular}

\subsection{2012}

\begin{tabular}{|rl|}
\hline
7DC & hexadezimal \\
3734 & oktal \\
11111011100 & bin�r \\
\hline
2012 & dezimal \\
\hline
\end{tabular}

\subsection{5,5625}

\begin{tabular}{|rl|}
\hline
5,9 & hexadezimal \\
5,44 & oktal \\
101,1001 & bin�r \\
\hline
5,5625 & dezimal \\
\hline
\end{tabular}

\subsubsection{Beispielrechnung f�r 5,5625}
Umwandlung in Bin�rsystem

		$375:2 = 187		$	Rest: 1 \\
		$187:2	= 93		$	Rest: 1 \\
		$93:2	= 46			$	Rest: 1 \\
		$46:2	= 23			$	Rest: 0 \\
		$23:2	= 11			$	Rest: 1 \\
		$11:2	= 5 			$	Rest: 1 \\
		$5:2		= 2 		$	Rest: 1 \\
		$2:2		= 1			$	Rest: 0 \\
		$1:2		= 0			$	Rest: 1 \\
		$= 101110111_2 = 375_{10} $ 
		
		
		$0,375\cdot2=0,75$ Ziffer: 0 \\
		$0,75\cdot2=1,5$	Ziffer: 1	\\
		$0,5\cdot2=1$	Ziffer: 1	\\
		
		$375,375_{10}=101110111,011_2$
			\\
			\\
		\\Umwandlung in Oktalsystem

		$ 375:8 = 46 $ Rest: 7\\
		$ 46 :8 = 5  $ Rest: 6 \\
		$ 5  :8 = 0  $ Rest: 5 \\
    $ = 567_8		 $

		$0,375\cdot8=3$ Ziffer : 3\\
		$=0,3_8$ 
		
		Daraus folgt: $375,375_{10}=567_8+0,3_8=567,3_8$\\
		\\
		Umwandlung in Hexadezimalsystem \\
		
		$375:16=23$	Rest:7 \\
		$23:16=1	$	Rest:7 \\
		$1:16=0		$	Rest:1 \\
		$=177_{16}$
		
		$0,375\cdot16=6$	Ziffer: 6 \\
	  $=0,6_{16}$
	  
	  $375,375_{10}=177_{16}+0,6_{16}=177,6_{16} $

\subsection{375,375}

\begin{tabular}{|rl|}
\hline
177,6 & hexadezimal \\
567,3 & oktal \\
101110111,011 & bin�r \\
\hline
375,375 & dezimal \\
\hline
\end{tabular}		
		
\section{Umwandlung in Dezimalzahlen } 
	
		
	\[1110,1001 = 2^{3}+2^{2}+2^{1}+2^{-1}+2^{-4} = 14,5625 \]
	\[10101,10011 = 2^{4}+2^{2}+2^{0}+2^{-1}+2^{-4}+2^{-5}	=	21,59375\]
		
\section{Addition im Dualsystem}

	\[ 25487_{10}+15190_{10}=40677\]
	\[ 25487_{10}=110001110001111_{2}=61617_{8}=638F_{16}\]
	\[ 15190_{10}=11101101010110_{2}=35526_{8}=3B56_{16}\]

	\begin{tabular}{r}
		0110001110001111\\
		+0011101101010110\\
		\hline
		$1001111011100101$\\
		$=117345_{8}=9EE5_{16}$
	\end{tabular}

\section{Multiplikation im Dualsystem}
	\begin{tabular}{r}
    
	  $10010011\cdot111001$\\
	  10010011\phantom{11001}\\
	  10010011\phantom{1001} \\
	  10010011\phantom{001} \\
	  00000000\phantom{01} \\
	  00000000\phantom{1} \\
	  10010011 \\
	  \hline
		111111111\phantom{00000}\\ \hline
	  10000010111011
    
	\end{tabular}

\section{Komplemente}

$K_{10} (4,582)_{10}= 10^2-4,582 = 95,4180               $\\
$K_{10-1} (0,1274)_{10} = 10^2-10^{-4} -0,1274 = 99,8725   $   \\          
$K_{2} (1,011)_{2} = 2^{2-1}-1,011 =4-1,375 = 2,625 = 10,101$\\
$K_{2-1}(100,01)_{2} = 16 - 0,125 - 4,25 = 11,625= 1011,101$\\

\section{Darstellung negativer Zahlen}

	1 = Ganzzahl im Dualsystem \\
	2 = Betrag und Vorzeichen\\
	3 = Exzess-127 Kodierung\\
	4 = Einerkomplement\\
	5 = Zweierkomplement\\

	\begin{tabular}{|l c c c c c|} 
		\hline
		Bin�r & 1 & 2 & 3 & 4 & 5  \\ \hline
		0000 1001 & 9 & 9 & -118 & 246 & 247\\ \hline
		0110 0101 & 101 & 101 & -26 & 154 & 155\\ \hline
		1000 0001 & 129 & -1 & 2 & -126 & -127\\ \hline
		1111 1011 & 251 & -123 & 124 & -4 & -5\\ \hline 
	\end{tabular}

\end{document}