\documentclass{article}
\usepackage[a4paper]{geometry}
\usepackage{amsfonts}
\usepackage{alltt}
\usepackage{amsmath,amssymb}        % Mathpack f�r Formeln jeder Art
\usepackage[parfill]{parskip}       % Autoamtisch Newline, wenn Zeilenumbruch im Quelltext.
\usepackage[utf8]{inputenc}         % UTF8 Zeichensatz. 
\usepackage{xstring}
\usepackage{tikz}                   %Wichtig f�r Zeichnungen aller Art
\input{kvmacros}                    %Kv Diagramme
\usepackage[siunitx]{circuitikz}    %Diagramme und Schaltungen

\title{RS - �bung 7}
\author{Arne Beer (MN 6489196), \\
Rafael Epplee (MN 6269560), \\
Julian Polatynski (MN 6424884)}

\begin{document}
\maketitle

\section*{7.1}

  \subsection*{a)}

  \begin{itemize}

  \item Kanonische Disjunktive Normalform:

  $f(x) = (\overline{x_3} \wedge x_2 \wedge x_1 ) \vee ( \overline{x_3} \wedge \overline{x_2} \wedge x_1 ) \vee ( \overline{x_3} \wedge \overline{x_2} \wedge \overline{x_1} ) \vee ( x_3 \wedge x_2 \wedge x_1 )$

  \item Kanonische Konjunktive Normalform:

  $f(x) = ( \overline{x_3} \vee x_2 \vee \overline{x_1}) \wedge ( \overline{x_3} \vee x_2 \vee x_1) \wedge ( \overline{x_3} \vee \overline{x_2} \vee x_1) \wedge (x_3 \vee \overline{x_2} \vee x_1$

  \item Reed Muller Form:

  $(x_3\vee \overline{x_2}\wedge (x_2 \vee \overline{x_1})$

  $\Leftrightarrow (x_3 \oplus x_2 \oplus 1 \oplus x_3 \oplus x_2 x3) \vee (x_2 \oplus x_1 \oplus 1 \oplus x_2 \oplus x_1 x_2$

  $\Leftrightarrow (1 \oplus x_2 \oplus x_2 x_3)\wedge(1 \oplus x_1 \oplus x_1 x_2)$

  \end{itemize}

  \subsection*{b)}

  \begin{itemize}

  \item Kanonische Disjunktive Normalform

  $f(x)= ( \overline{x_3}\wedge x_2 \wedge x_1)\vee ( \overline{x_3} \wedge \overline{x_2} \wedge x_1)\vee (x_3 \wedge \overline{x_2} \wedge \overline{x_1} )\vee (x_3 \wedge x_2 \wedge \overline{x_1}) $

  \item Kanonische Konjunktive Normalform

  $f(x)= ( \overline{x_3} \vee \overline{x_2} \vee \overline{x_1})\wedge (x_3 \vee x_2 \vee x_1)\wedge (x_3 \vee \overline{x_2} \vee x_1)\wedge ( \overline{x_3} \vee x_2 \vee \overline{x_1}) $

  \item Reed Muller Form:

  $\overline{x_3} \oplus \overline{x_1} $

  $\Leftrightarrow x_3 \oplus 1 \oplus x_1 \oplus 1 = x_3 \oplus x_1$

  \end{itemize}

\section*{7.2}
    \subsection*{a)}
        \begin{itemize}
            \item NAND(a, a) ergibt immer 0, falls a = 1. Falls a = 0 ist, ergibt es 0. Dementsprechend ist NOT(a) = NAND(a, a).
            
            \item NOT(NAND(a,b)) ergibt immer das Gegenteil von NAND, also AND:
            NOT(NAND(a,b)) = NAND(NAND(a,b), NAND(a,b)).
            
            \item Wenn man die Eingangsvariablen invertiert und dann NAND(a,b) nimmt, erh�lt man f�r jeden Wert ausser (0,0) eine 1, was eine OR-Operation repr�sentiert.
            OR(a,b) = NAND(NOT(a), NOT(b))
            \end{itemize}


  \subsection*{b)}
      \begin{itemize}

      \item$f(x_3,x_2,x_1)=(\overline{x_3}(\overline{x_2}\vee x_1)\wedge (x_1(\overline{x_2 \vee x_1}))$

      $\Leftrightarrow \overline{x_3}\overline{x_2}\vee\overline{x_3}x_1 \wedge x_1\overline{x_2}\vee x_1x_1$
      
      $\Leftrightarrow \overline{x_3}\wedge \overline{x_2} \vee x_1$

      $\Leftrightarrow \overline{(\overline{\overline{x_3}\wedge \overline{x_2}}\wedge \overline{x_1})}$

      $\Leftrightarrow \text{NAND}(\text{NAND}(\text{NAND}(x_3,x_3),\text{NAND}(x_2,x_2)), \text{NAND} (x_1,x_1))$

      \end{itemize}

\section*{7.3}
  \subsection*{a)}

    \begin{tabular}{c c c c c|c }
     A & $ x_3$ & $x_2$ & $x_1$ & $x_0$ & $f(x_0, x_1, x_2, x_3)$ \\ \hline
    0&0&0&0&0&1\\
    1&0&0&0&1&0\\
    2&0&0&1&0&1\\
    3&0&0&1&1&1\\
    4&0&1&0&0&0\\
    5&0&1&0&1&1\\
    6&0&1&1&0&1\\
    7&0&1&1&1&1\\
    8&1&0&0&0&1\\
    9&1&0&0&1&1\\


    \end{tabular}

    \begin{tabular}{c c c c c|c }
    B & $ x_3$ & $x_2$ & $x_1$ & $x_0$ & $f(x_0, x_1, x_2, x_3)$ \\ \hline
    0&0&0&0&0&1\\
    1&0&0&0&1&1\\
    2&0&0&1&0&1\\
    3&0&0&1&1&1\\
    4&0&1&0&0&1\\
    5&0&1&0&1&0\\
    6&0&1&1&0&0\\
    7&0&1&1&1&1\\
    8&1&0&0&0&1\\
    9&1&0&0&1&1\\

    \end{tabular}

    \subsection{b)}

    KV-Diagramm f�r A:

    \kvnoindex
    \karnaughmap{4}{$f(a,b,c,d):$}{{$d$}{$b$}{$c$}{$a$}}%
    {1001111111******}
    {
    \put(3,2){\oval(1.9,3.9)[]}
    \put(2,1){\oval(3.9,1.9)[]}
    \put(2,2){\oval(1.9,1.9)[]}
    \put(0.5,4){\oval(0.9,1.9)[br]}%Ecke oben links
    \put(0,3.5){\oval(1.9,0.9)[rb]}%
    \put(0,0,5){\oval(1.9,0.9)[tr]}%Ecke unten links
    \put(0.5,0){\oval(0.9,1.9)[rt]}
    \put(3.5,4){\oval(0.9,1.9)[bl]}%Ecke oben rechts
    \put(4,3.5){\oval(1.9,0.9)[lb]}
    \put(3.5,0){\oval(0.9,1.9)[tl]}%Ecke unten rechts
    \put(4,0.5){\oval(1.9,0.9)[lt]}
    }

    Nach den ausgew�hlten Schleifen ergibt sich f�r Sektion A:

    \[f(a,b,c,d) = (a \wedge c) \vee (b) \vee (d) \vee (\overline{a} \wedge \overline{c})\]

    KV-Diagramm f�r B:

    \karnaughmap{4}{$f(a,b,c,d):$}{{$d$}{$b$}{$c$}{$a$ }}%
    {1110110111******}
    {
    \put(2,0){\oval(3.9,1.9)[t]}
    \put(2,4){\oval(3.9,1.9)[b]}
    \put(0.5,2){\oval(0.9,3.9)[]}
    \put(2.5,2){\oval(0.9,3.9)[]}
    }
    
    Hier ergibt sich in der disjunktiven Normalform f�r Sektion B:

    \[ f(a,b,c,d) = (\overline{c}) \vee (\overline{a} \wedge \overline{b}) \vee (a \wedge b)\]

\section*{7.4}
  \subsection*{a)}

    Es wird eine 1 angegeben, wenn die Leistungsaufnahme gr��er als 6 ist.

     \begin{tabular}{c c c c|c }
    $ x_3$ & $x_2$ & $x_1$ & $x_0$ & $f(x_0, x_1, x_2, x_3)$ \\ \hline
    0&0&0&0&0\\
    0&0&0&1&0\\
    0&0&1&0&0\\
    0&0&1&1&0\\
    0&1&0&0&0\\
    0&1&0&1&1\\
    0&1&1&0&0\\
    0&1&1&1&1\\
    1&0&0&0&0\\
    1&0&0&1&0\\
    1&0&1&0&0\\
    1&0&1&1&0\\
    1&1&0&0&1\\
    1&1&0&1&1\\
    1&1&1&0&1\\
    1&1&1&1&1\\

    \end{tabular}

  \subsection*{b, c)}
    KV-Diagramm mit gr��tm�glichen Schleifen:

    \karnaughmap{4}{f($x_0$,$x_1$,$x_2$,$x_3$):}{{$x_3$}{$x_1$}{$x_2$}{$x_0$}}%
    {0001000100110011}
    {
    \put(2,2){\oval(1.9,1.9)[]}
    \put(2,1.5){\oval(3.9,0.9)[]}
    }

    Die Dazugeh�rige Formel in disjunktiver Form:

    \[ f(x_0, x_1, x_2, x_3)=(x_2 \wedge x_3) \vee (x_0 \wedge x_2)\]

  \subsection*{d)}
  Im Prinzip sagt die Funktion nichts anderes als "Entweder $x_0$ und $x_1$, oder $x_2$ und $x_3$". Das l�sst sich ganz einfach in ein Schaltnetz �bertragen.
  
    \begin{circuitikz} \draw
      (0,2) node[and port] (myand) {}
      (2,1) node[or port] (myor) {}
      (0,0) node[and port] (myand2) {}
      (myand.in 1) node[anchor=east] {$x_0$}
      (myand.in 2) node[anchor=east] {$x_1$}
      (myand2.in 1) node[anchor=east] {$x_2$}
      (myand2.in 2) node[anchor=east] {$x_3$}
      (myand.out) -- (myor.in 1)
      (myand2.out) -- (myor.in 2)
    ;\end{circuitikz}

\end{document}