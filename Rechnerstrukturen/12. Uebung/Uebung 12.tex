\documentclass{article}

\usepackage[utf8]{inputenc}
\usepackage{alltt}

\newcommand\code[1]{\begin{alltt}#1\end{alltt}}

\title{RS-Übung 12}
\author{Arne Beer \\ Rafael Epplee \\ Florian Tobergte}

\begin{document}

\maketitle

\section*{Aufgabe 1}

    \subsection*{a)}
        \code{0x0000CAFE + 0x00000004 = 0x0000CB02}

    \subsection*{b)}
        \code{0x000000CB - 0x0000000C = 0x000000BF}

    \subsection*{c)}
        \code{0x00012300 * 0x00000020 = 0x00246000}

    \subsection*{d)}
        \code{0x00000042 + 0x00000001 = 0x00000043}

    \subsection*{e)}
        \code{0x0000000C - 0x00000001 = 0x0000000B}

    \subsection*{f)}
        \code{0x00000100 - 0x00000004 = 0x000000FC}

\section*{Aufgabe 2}
    
    Indem man ein Register von sich selbst subtrahiert, ergibt sich immer der Wert 0:

    \code{subl \%eax, \%eax}

\section*{Aufgabe 3}

\section*{Aufgabe 4}

    \begin{alltt} 
        pushl %ebp
        subl $32, %ebp
        imull $142, %ebp
        shrl $2, %ebp
    \end{alltt}

\end{document}