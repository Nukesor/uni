\documentclass{article}
\usepackage{amsfonts}
\usepackage[a4paper]{geometry}
\usepackage{alltt}
\usepackage{lmodern}
\usepackage{amssymb}
\usepackage{mathtools}
\usepackage{amsmath}
\usepackage{enumerate}
\usepackage{array}
\usepackage{listings}
\usepackage{fullpage}
\usepackage[parfill]{parskip}
\usepackage[utf8]{inputenc}
\usepackage[ngerman]{babel} 
\usepackage{graphicx}
\usepackage{tikz}
\usepackage{fancyhdr}
\usepackage{pgfplots}
\usepackage{multicol}

\usetikzlibrary{arrows,automata}

\tikzstyle{help lines}=[blue!50,very thin]
\tikzstyle{help lines}+=[dashed]
\tikzstyle{Kreis}= [circle,draw]


\title{GDB Uebung 2, Gruppe 61}
\author{Arne Beer, MN 6489196\\
        Oliver Heidmann, MN 6420331,
        Minh Nguyen, MN 6423136}

\begin{document}
    \maketitle
    \begin{enumerate}
        \item %1
            \begin{enumerate}
                \item $ \pi_{\text{Sorte}}((\sigma_{\text{Vorname="Horst''}} \text{Personen})\bowtie_{\text{Entdecker=PNR}} \text{Obst})$ 
                \item $ \pi_{\text{Vorname, Nachname}}((\sigma_{\text{Symptom="Halskratzen''}}\text{Allergien})\bowtie \text{Person})$
                \item $ \pi_{\text{Sorte, Nachname}}(((\sigma_{\text{Symptom="Wuergreiz''}}\text{Allergie})\bowtie \text{Person})\bowtie_{\text{Entdecker=PNR}}\text{Obst})$
            \end{enumerate}
        \item 
            \begin{enumerate}
                \item %2a
                    \begin{verbatim}
CREATE TABLE Rennfahrer(
    RID int PRIMARY KEY,
    Vorname varchar(50) NOT NULL,
    Nachname varchar(50) NOT NULL,
    Geburt varchar(10) NOT NULL,
    Wohnort varchar(50),
    CONSTRAINT Rennstall FOREIGN KEY (Rennstall) PREFERENCES Rennstall (RSID)
);
CREATE TABLE Rennstall(
    RSID int PRIMARY KEY,
    Name varchar(50) NOT NULL
    Teamchef varchar(50),
    Budget int check(0<Budget AND Budget<500)
);
CREATE TABLE Rennort(
    OID int PRIMARY KEY,
    Name varchar(50) NOT NULL,
    Strecke varchar(50) NOT NULL,
);
CREATE TABLE Platzierung(
    CONSTRAINT RID FOREIGN KEY (Rennfahrer) PREFERENCES Rennfahrer(PID),
    CONSTRAINT OID FOREIGN KEY (Rennort) PREFERENCES Rennort(OID),
    Platz int NOT NULL,
    CONSTRAINT pk_platz PRIMARY KEY (RID, OID),
)
                    \end{verbatim}
                \item
                    Wenn eine Table auf eine andere Table refernziert, welche noch nicht initianilisiert ist, wird ein Fehler geworfen. Daher muss die Fremdschlüssel im Nachträglich hinzugefügt werden.
                \item
                    \begin{verbatim}
ALTER TABLE Rennstall(
    ADD CONSTRAINT Star FOREIGN KEY(Rennfahrer) PREFERENCES Rennfahrer(PID)
)

                    \end{verbatim}
                \item
                    \begin{enumerate}
                        \item
                            \begin{verbatim}
delete from Rennfahrer S
where S.Vorname Like "F%"      
                            \end{verbatim}
                        \item
                            \begin{verbatim}
drop table *                             
                            \end{verbatim}
                    \end{enumerate}
            \end{enumerate}
        \item %3
            \begin{enumerate}
                \item 
                    \begin{verbatim}
SELECT DISTINCT B.Sorte
FROM Person A, Obst B, Allergie C
WHERE A.Vorname = "Peter" AND A.Nachname = "Meyer"
AND C.PNR = A.PNR   
AND C.ONR = B.ONR
ORDER BY B.Sorte DESC;
                    \end{verbatim}
                \item
                    \begin{verbatim}
SELECT A.PNR, A.Nachname,COUNT (DISTINCT B.PNR)
FROM  Person A, Allergie B
WHERE B.PNR = A.PNR;
                    \end{verbatim}

                \item
                    \begin{verbatim}
SELECT A.PNR,
FROM  Person A, Obst B
WHERE B.Entdecker = A.PNR AND COUNT(B.Sorte) > 6;
                    \end{verbatim}

                \item
                    \begin{verbatim}
SELECT A.Vorname,A.Nachname
FROM Person A, Person B, Obst C
WHERE C.Entdecker = B.PNR
    AND B.Vorname = A.Vorname;
                    \end{verbatim}

                \item
                    \begin{verbatim}
SELECT A.Vorname, A.Nachname
FROM PSERON  A
WHERE NOT EXISTS
    (SELECT A.PNR
    FROM OBST.B
    WHERE A.PNR = B.Entdecker);
                    \end{verbatim}
                \end{enumerate}

        \item % 4
            Card(Person) = 2000,\\
            Card(Obst) = 25, \\
            400 verschiedene Nachnamen, \\
            Sorte eindeutig, \\
            Obstsorten sind eindeutig und genau 5 beginnen mit dem Buchstaben K , \\  

            \begin{tikzpicture}[style={sibling distance=4cm}]
                \node{}
                    child{node{$\pi_{\text{DISTINCT PNR, Vorname, Nachname}}$}
                            child{node{$\sigma_{\text{Lieblingsobst = ONR $ \wedge $ Sorte LIKE ''K\%''}}$}
                                child{node{Person}}
                                child{node{Obst}}
                            }
                    }
                ;
                        
            \end{tikzpicture} \\

            \newpage
            Optimierter Baum:

            \begin{tikzpicture}[>=stealth',shorten >=1pt,auto,node distance=4cm, semithick]
                    %nodes
                    \node[]     (A)                {};                 
                    \node[]     (B) [below of = A] {$\pi_{\text{DISTINCT PNR, Vorname, Nachname}}$};
                    \node[]     (C) [below of = B] {$\bowtie_{\text{Lieblingsobst = ONR}}$};
                    \node[]     (D) [below left of = C] {$\sigma{\text{DISTINCT PNR, Vorname, Nachnamen}}$};
                    \node[]     (E) [below of = D] {Person};
                    \node[]     (F) [below right of = C] {$ \sigma_{\text{Sorte LIKE ''K\%''}}$};
                    \node[]     (G) [below of = F] {Obst};

                    
                    %initial graphic
                    \path[]               (A) edge   node{5 Tupel, 3 Attribute}             (B);
                    \path[]               (B) edge   node{(3) 5 Tupel, 9 Attribute}             (C);
                    \path[]               (C) edge   node{(1) 5 Tupel, 5 Attribute}             (D);
                    \path[]               (C) edge   node{(2) 5 Tupel, 4 Attribute}             (F);
                    \path[]               (D) edge   node{2000 Tupel, 5 Attribute}             (E);
                    \path[]               (F) edge   node{25 Tupel, 4 Attribute}             (G);
            \end{tikzpicture} \\
            (1) $\text{SF}(\text{P}_{\text{DISTINCT PNR, Vorname, Nachname}}) = \frac{1}{400}$ \\
            (2) $\text{SF}(\text{P}_{\text{Sorte LIKE ''K\%''}})$ \\
            (3) $\text{Card}(\text{Person}\bowtie \text{Obst}) = \text{Card(Person)}$\\

            Endergebnis ist optimaler, da die Selections frueh angewendet werden und join genutzt wird, wodurch die Zwischenergebnisse schon zu Beginn sehr klein werden.

    \end{enumerate}
\end{document}b